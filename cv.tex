\documentclass[a4paper]{my_cv}
\begin{document}

%\addtolength{\bottommargin}{-5cm}
\thispagestyle{empty}	
\pageheader{Nikola~}{Milev}
	{+381653509370}
 	{\href{mailto:nikola.n.milev@gmail.com}{nikola.n.milev@\\gmail.com}}
	{ \href{https://github.com/NikolaMilev}{/NikolaMilev}} 	
 	{ \href{https://www.linkedin.com/in/nikola-n-milev/}{/in/nikola-n-milev/}}
	{ Otona Zupancica 1/34, Belgrade }
	{12.05.1993.}

% TODO
% izvali kako da promenis font na neki koji je kul (pogledaj ove tempate?) DONE
% ubaci fotku onu sa strane (vidi kako je u template i da li tikz moze to da uradi) DONE
% izvali kako da podesis kolonu sa strane gde mozes da stavljas stvari (opet, vidi template) DONE
% vidi kako se ukljucuju progress barovi DONE
% vidi kako u sekciji za obrazovanje da ga nateras da bude minimalne sirine te i te
% itd (nadji jos kul ideja)

 
\begin{aside}
\section{Languages}
% 3 is max
\bodyfont\skills{{Serbian/3}, {English/3}}
~
~
\section{Programming}
% 3 is max
\skills{{Eclipse/2}, {Intellij IDEA/2}, {Git/2}, {LaTeX/2}, {GNU Linux/3},{Haskell/2},{Python/2}, {C++/2},{Java/3}, {C/3}}
\end {aside}
~
~\\
\section{Experience}
\begin{entrylist}
\entry
    {10/2017~\textemdash \\4/2018}
    {Software developer}
    {RT-RK}
    {Worked on maintaining an Android mobile application for facial recognition and emotion detection.}
\entry
    {10/2016~\textemdash \\present}
    {Teaching Associate, Department of Computer Science}
    {Faculty of Mathematics, University of Belgrade}
    {Teaching courses, organising and grading exams. Courses taught: Introduction to programming, Introduction to object oriented programming and Introduction to computer organisation and architecture.}
\entry
    {07/2016~\textemdash \\10/2016}
    {Intern}
    {ESDL (Electronics Systems Design Limited), Malta}
    {Implemented a RaspberryPi server with UART communication. Implemented in C, using OpenSSL.}
\entry
    {05/2016}
    {Intern}
    {sTech d.o.o. Belgrade, member of UNIQA Group Austria}
    {Worked within three teams in order to get introduced to the system used for processing insurance policies. }
\end{entrylist}
\section{Education}
\begin{entrylist}
  \entry
    {2016~\textemdash \\present}
    {Master's Degree in Computer Science}
    {Faculty of Mathematics, University of Belgrade}
    {Currently learning about machine learning, functional programming, automated reasoning, etc. GPA 9.6 out of 10.}
  \entry
    {2012~\textemdash \\2016}
    {Bachelor's Degree in Computer Science}
    {Faculty of Mathematics, University of Belgrade}
    {Passed many courses that covered important topics such as algorithms, object oriented programming, Unix system programming, etc. Graduated as one of the best students in the generation. GPA 9.61 out of 10.}
  \entry
    {2008~\textemdash \\2012}
    {High School}
    {Grammar School, Valjevo}
    {Finished with several awards for good students. Was a member of the school choir and took part in various music manifestations.}
\end{entrylist}

\section{Awards and activities}
\bodyfont
\begin{aplist}
\apentry{2016}{Dositeja scholarship: a scholarship awarded to 800 best students of undergraduate studies in Serbia.}
\apentry{10/2016}{Brand New Engineers Hackathon, team Schwifty, 3rd place.}
\end{aplist}
%\section{Projects}
%\bodyfont
%\begin{aplist}
%\apentry{Origami \\simulator}{A simulation for origami representation in 3D, with basic paper folding allowed. (a team project written in C++ using STL and Qt libraries; implemented the data structure for an origami figure, serialization and several smaller tasks)}
%\end{aplist}

\section{Projects}
\setlength{\tabcolsep}{6pt}
\begin{tabularx}{1.07\textwidth}{XX}
\textbf{Minipascal to flowchart:} A small program that compiles a small subset of Pascal into a LaTeX flowchart, written in C++ using Flex and Bison. &
\textbf{Origami simulator:} A group project written in C++ using STL and Qt libraries. Implemented the data structure, serialization and several smaller tasks. \\
\textbf{Turing machine:} A Turing machine simulation, written in Java. &
\textbf{Minesweeper:} An implementation of the game written in Java. \\
\end{tabularx}
~
\section{Interests}
\bodyfont
\begin{flushleft} 
\begin{tabularx}{\textwidth}{XXX}

\faBicycle ~Cycling &
\faLifeRing ~~Swimming &
\faUniversity ~~Sightseeing \\
\faMusic ~~~Music &
\faPlane ~~~Travelling &
\faCutlery ~~~Cooking \\
\faKeyboardO ~~Programming &
\faGamepad ~~Video games &
\faTelevision ~~Movies and TV

\end{tabularx}
\end{flushleft}

\end{document} 